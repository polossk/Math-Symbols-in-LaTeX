% pdf 设置区
%---------------------------------纸张大小设置---------------------------------%
\usepackage{geometry}
\geometry{a4paper,left=1.91cm,right=1.91cm,top=2.05cm,bottom=2.05cm}
%------------------------------------------------------------------------------%


%----------------------------------必要库支持----------------------------------%
\usepackage{xcolor}
\usepackage{tikz}
\usepackage{layouts}
\usepackage{clrscode}
%------------------------------------------------------------------------------%


%--------------------------------添加书签超链接--------------------------------%
\usepackage[unicode=true,colorlinks=false,pdfborder={0 0 0}]{hyperref}
    % 在此处修改打开文件操作
    \hypersetup{
        bookmarks=true,         % show bookmarks bar?
        bookmarksopen=true,     % expanded all bookmark?
        pdftoolbar=true,        % show Acrobat’s toolbar?
        pdfmenubar=true,        % show Acrobat’s menu?
        pdffitwindow=true,      % window fit to page when opened
        pdfstartview={FitH},    % fits the width of the page to the window
        pdfnewwindow=true,      % links in new PDF window
    }
    % 在此处添加文章基础信息
    \hypersetup{
        pdftitle={Math Symbols in LaTeX Manual},
        pdfauthor={polossk},
        pdfsubject={LaTeX Manual},
        pdfcreator={XeLaTeX},
        pdfproducer={XeLaTeX},
        pdfkeywords={LaTeX, Math Symbols},
    }
%------------------------------------------------------------------------------%


%------------------------------添加插图与表格控制------------------------------%
\usepackage{graphicx}
\usepackage[font=small,labelsep=quad]{caption}
\usepackage{booktabs}
\usepackage{tabularx}
\usepackage{setspace}
%------------------------------------------------------------------------------%


%---------------------------------添加列表控制---------------------------------%
\usepackage{enumerate}
\usepackage{enumitem}
%------------------------------------------------------------------------------%


%---------------------------------设置页眉页脚---------------------------------%
\usepackage{fancyhdr}
\usepackage{fancyref}
\pagestyle{fancy}
\lhead{}
\rhead{}
\chead{}
\lfoot{Version: v\artversion}
\cfoot{\thepage}
\rfoot{by \href{https://github.com/polossk}{polossk}}
\renewcommand{\headrulewidth}{0.4pt}
\renewcommand{\headwidth}{\textwidth}
\renewcommand{\footrulewidth}{0pt}
%------------------------------------------------------------------------------%


%-------------------------------数学特殊符号控制-------------------------------%
\usepackage{math-symbols}
%------------------------------------------------------------------------------%


%---------------------------------添加代码控制---------------------------------%
\colorlet{lightgray}{gray!40}
\colorlet{darkred}{red!70!black}
\usepackage{listings}
\lstset{
    basicstyle=\color{darkred}\normalsize\ttfamily,
    backgroundcolor=\color{lightgray},
    breaklines=true,
}
%------------------------------------------------------------------------------%

\endinput
% 这是简单的 article 的导言区设置,不能单独编译。