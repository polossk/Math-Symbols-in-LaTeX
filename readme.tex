\documentclass{article}
% pdf 设置区
%---------------------------------纸张大小设置---------------------------------%
\usepackage{geometry}
\geometry{a4paper,left=1.91cm,right=1.91cm,top=2.05cm,bottom=2.05cm}
%------------------------------------------------------------------------------%


%--------------------------------添加书签超链接--------------------------------%
\usepackage[unicode=true,colorlinks=false,pdfborder={0 0 0}]{hyperref}
% 在此处修改打开文件操作
\hypersetup{
    bookmarks=true,         % show bookmarks bar?
    bookmarksopen=true,     % expanded all bookmark?
    pdftoolbar=true,        % show Acrobat’s toolbar?
    pdfmenubar=true,        % show Acrobat’s menu?
    pdffitwindow=true,      % window fit to page when opened
    pdfstartview={FitH},    % fits the width of the page to the window
    pdfnewwindow=true,      % links in new PDF window
}
% 在此处添加文章基础信息
\hypersetup{
    pdftitle={Math Symbols in LaTeX Manual},
    pdfauthor={polossk},
    pdfsubject={LaTeX Manual},
    pdfcreator={XeLaTeX},
    pdfproducer={XeLaTeX},
    pdfkeywords={LaTeX, Math Symbols},
}
%------------------------------------------------------------------------------%


%------------------------------添加插图与表格控制------------------------------%
\usepackage[font=small,labelsep=quad]{caption}
\usepackage{booktabs}
\usepackage{setspace}
%------------------------------------------------------------------------------%


%---------------------------------添加列表控制---------------------------------%
\usepackage{enumerate}
\usepackage{enumitem}
%------------------------------------------------------------------------------%


%---------------------------------设置页眉页脚---------------------------------%
\usepackage{fancyhdr}
\usepackage{fancyref}
\pagestyle{fancy}
\lhead{}
\rhead{}
\chead{}
\lfoot{Version: v\artversion}
\cfoot{\thepage}
\rfoot{by \href{https://github.com/polossk}{polossk}}
\renewcommand{\headrulewidth}{0.4pt}
\renewcommand{\headwidth}{\textwidth}
\renewcommand{\footrulewidth}{0pt}
%------------------------------------------------------------------------------%


%-------------------------------数学特殊符号控制-------------------------------%
\usepackage{math-symbols}
%------------------------------------------------------------------------------%


%---------------------------------添加代码控制---------------------------------%
\usepackage{xcolor}
\colorlet{lightgray}{gray!40}
\colorlet{darkred}{red!70!black}
\usepackage{listings}
\lstset{
    basicstyle=\color{darkred}\normalsize\ttfamily,
    backgroundcolor=\color{lightgray},
    breaklines=true,
}
%------------------------------------------------------------------------------%

\endinput
% 这是简单的 article 的导言区设置,不能单独编译。
%------------------------------------------------------------------------------%
\newcommand\artversion{2.2.3.1026}
\title{Math-Symbols-in-\LaTeX{}-Manual}
\author{polossk}
\date{Version: v\artversion, Last Update: \today}
%------------------------------------------------------------------------------%
\begin{document}
%------------------------------------------------------------------------------%
\maketitle

Add \lstinline`\usepackage{math-symbols}` in your document's preamble.

And you will no longer need use other math package in most instances.

% \tableofcontents
\thispagestyle{fancy}
\renewcommand{\baselinestretch}{1.25}
%------------------------------------------------------------------------------%


%------------------------------------------------------------------------------%
\section{Constants and Useful Symbols}
%------------------------------------------------------------------------------%
\begin{tabular}{*{10}{l}}
    $\mi$    & \lstinline`\mi`  & $\mnatr$ & \lstinline`\mnatr`  & $\mcmpx$  & \lstinline`\mcmpx`  & $\mscab$      & \lstinline`\mscab`  & $\mslbg[{[a, b]}]{m}$ & \lstinline`\mslbg[{[a, b]}]{m}`  \\
    $\mj$    & \lstinline`\mj`  & $\mintg$ & \lstinline`\mintg`  & $\mhilb$  & \lstinline`\mhilb`  & $\mscon{(I)}$ & \lstinline`\mscon{(I)}` & $\mssbl[{[a, b]}]{m}$ & \lstinline`\mssbl[{[a, b]}]{m}` \\
    $\me$    & \lstinline`\me` & $\mrato$ & \lstinline`\mrato` & $\mcond$  & \lstinline`\mcond` & $\mslbg{2}$   & \lstinline`\mslbg{2}` &                                                 \\
    $1\mdeg$ & \lstinline`1\mdeg` & $\mreal$ & \lstinline`\mreal` & $\mconst$ & \lstinline`\mconst` & $\mssbl{2}$   & \lstinline`\mssbl{2}` &                                                 \\
\end{tabular}
%------------------------------------------------------------------------------%


%------------------------------------------------------------------------------%
\section{Vector and Matrix Defination}
%------------------------------------------------------------------------------%


%------------------------------------------------------------------------------%
\subsection{Vector Notations}
%------------------------------------------------------------------------------%
Use \lstinline`\mv<name>` as the abbr of ``\underline{M}ath \underline{V}ector''.

\begin{tabular}{*{12}{l}}
    $\mva$ & \lstinline`\mva` & $\mvk$ & \lstinline`\mvk` & $\mvu$ & \lstinline`\mvu` & $\mvalpha$   & \lstinline`\mvalpha` & $\mvlambda$  & \lstinline`\mvlambda` & $\mvchi$        & \lstinline`\mvchi` \\
    $\mvb$ & \lstinline`\mvb` & $\mvl$ & \lstinline`\mvl` & $\mvv$ & \lstinline`\mvv` & $\mvbeta$    & \lstinline`\mvbeta` & $\mvmu$      & \lstinline`\mvmu` & $\mvpsi$        & \lstinline`\mvpsi` \\
    $\mvc$ & \lstinline`\mvc` & $\mvm$ & \lstinline`\mvm` & $\mvw$ & \lstinline`\mvw` & $\mvgamma$   & \lstinline`\mvgamma` & $\mvnu$      & \lstinline`\mvnu` & $\mvomega$      & \lstinline`\mvomega` \\
    $\mvd$ & \lstinline`\mvd` & $\mvn$ & \lstinline`\mvn` & $\mvx$ & \lstinline`\mvx` & $\mvdelta$   & \lstinline`\mvdelta` & $\mvxi$      & \lstinline`\mvxi` & $\mvvarepsilon$ & \lstinline`\mvvarepsilon` \\
    $\mve$ & \lstinline`\mve` & $\mvo$ & \lstinline`\mvo` & $\mvy$ & \lstinline`\mvy` & $\mvepsilon$ & \lstinline`\mvepsilon` & $\mvpi$      & \lstinline`\mvpi` & $\mvvarkappa$   & \lstinline`\mvvarkappa` \\
    $\mvf$ & \lstinline`\mvf` & $\mvp$ & \lstinline`\mvp` & $\mvz$ & \lstinline`\mvz` & $\mvzeta$    & \lstinline`\mvzeta` & $\mvrho$     & \lstinline`\mvrho` & $\mvvarphi$     & \lstinline`\mvvarphi` \\
    $\mvg$ & \lstinline`\mvg` & $\mvq$ & \lstinline`\mvq` &        &                         & $\mveta$     & \lstinline`\mveta` & $\mvsigma$   & \lstinline`\mvsigma` & $\mvvarpi$      & \lstinline`\mvvarpi` \\
    $\mvh$ & \lstinline`\mvh` & $\mvr$ & \lstinline`\mvr` &        &                         & $\mvtheta$   & \lstinline`\mvtheta` & $\mvtau$     & \lstinline`\mvtau` & $\mvvarrho$     & \lstinline`\mvvarrho` \\
    $\mvi$ & \lstinline`\mvi` & $\mvs$ & \lstinline`\mvs` &        &                         & $\mviota$    & \lstinline`\mviota` & $\mvupsilon$ & \lstinline`\mvupsilon` & $\mvvartheta$   & \lstinline`\mvvartheta` \\
    $\mvj$ & \lstinline`\mvj` & $\mvt$ & \lstinline`\mvt` &        &                         & $\mvkappa$   & \lstinline`\mvkappa` & $\mvphi$     & \lstinline`\mvphi` &                                           \\
\end{tabular}
%------------------------------------------------------------------------------%


%------------------------------------------------------------------------------%
\subsection{Matrix Notations}
%------------------------------------------------------------------------------%
Use \lstinline`\mm<name>` as the abbr of ``\underline{M}ath \underline{M}atrix''.

\begin{tabular}{*{14}{l}}
    $\mma$ & \lstinline`\mma`  & $\mmg$ & \lstinline`\mmg`  & $\mmm$ & \lstinline`\mmm`  & $\mms$ & \lstinline`\mms`  & $\mmy$ & \lstinline`\mmy` & $\mmgamma$  & \lstinline`\mmgamma`  & $\mmsigma$   & \lstinline`\mmsigma`  \\
    $\mmb$ & \lstinline`\mmb`  & $\mmh$ & \lstinline`\mmh`  & $\mmn$ & \lstinline`\mmn`  & $\mmt$ & \lstinline`\mmt`  & $\mmz$ & \lstinline`\mmz` & $\mmdelta$  & \lstinline`\mmdelta`  & $\mmupsilon$ & \lstinline`\mmupsilon`  \\
    $\mmc$ & \lstinline`\mmc`  & $\mmi$ & \lstinline`\mmi`  & $\mmo$ & \lstinline`\mmo`  & $\mmu$ & \lstinline`\mmu`  &        &                         & $\mmtheta$  & \lstinline`\mmtheta`  & $\mmphi$     & \lstinline`\mmphi`  \\
    $\mmd$ & \lstinline`\mmd`  & $\mmj$ & \lstinline`\mmj`  & $\mmp$ & \lstinline`\mmp`  & $\mmv$ & \lstinline`\mmv` &        &                         & $\mmlambda$ & \lstinline`\mmlambda` & $\mmpsi$     & \lstinline`\mmpsi` \\
    $\mme$ & \lstinline`\mme` & $\mmk$ & \lstinline`\mmk` & $\mmq$ & \lstinline`\mmq` & $\mmw$ & \lstinline`\mmw` &        &                         & $\mmxi$     & \lstinline`\mmxi` & $\mmomega$   & \lstinline`\mmomega` \\
    $\mmf$ & \lstinline`\mmf` & $\mml$ & \lstinline`\mml` & $\mmr$ & \lstinline`\mmr` & $\mmx$ & \lstinline`\mmx` &        &                         & $\mmpi$     & \lstinline`\mmpi` &                                         \\
\end{tabular}
%------------------------------------------------------------------------------%


%------------------------------------------------------------------------------%
\subsection{Tensor Notations}
%------------------------------------------------------------------------------%
Use \lstinline`\mt<name>` as the abbr of ``\underline{M}ath \underline{T}ensor''.

\begin{tabular}{*{14}{l}}
    $\mta$ & \lstinline`\mta` & $\mtg$ & \lstinline`\mtg` & $\mtm$ & \lstinline`\mtm` & $\mts$ & \lstinline`\mts` & $\mty$ & \lstinline`\mty` & $\mtgamma$  & \lstinline`\mtgamma` & $\mtsigma$   & \lstinline`\mtsigma` \\
    $\mtb$ & \lstinline`\mtb` & $\mth$ & \lstinline`\mth` & $\mtn$ & \lstinline`\mtn` & $\mtt$ & \lstinline`\mtt` & $\mtz$ & \lstinline`\mtz` & $\mtdelta$  & \lstinline`\mtdelta` & $\mtupsilon$ & \lstinline`\mtupsilon` \\
    $\mtc$ & \lstinline`\mtc` & $\mti$ & \lstinline`\mti` & $\mto$ & \lstinline`\mto` & $\mtu$ & \lstinline`\mtu` &        &                          & $\mttheta$  & \lstinline`\mttheta` & $\mtphi$     & \lstinline`\mtphi` \\
    $\mtd$ & \lstinline`\mtd` & $\mtj$ & \lstinline`\mtj` & $\mtp$ & \lstinline`\mtp` & $\mtv$ & \lstinline`\mtv` &        &                          & $\mtlambda$ & \lstinline`\mtlambda` & $\mtpsi$     & \lstinline`\mtpsi` \\
    $\mte$ & \lstinline`\mte` & $\mtk$ & \lstinline`\mtk` & $\mtq$ & \lstinline`\mtq` & $\mtw$ & \lstinline`\mtw` &        &                          & $\mtxi$     & \lstinline`\mtxi` & $\mtomega$   & \lstinline`\mtomega` \\
    $\mtf$ & \lstinline`\mtf` & $\mtl$ & \lstinline`\mtl` & $\mtr$ & \lstinline`\mtr` & $\mtx$ & \lstinline`\mtx` &        &                          & $\mtpi$     & \lstinline`\mtpi` &                                         \\
\end{tabular}
%------------------------------------------------------------------------------%


%------------------------------------------------------------------------------%
\subsection{Transposed Matrix Notations}
%------------------------------------------------------------------------------%
Use \lstinline`\mm<name>t` as the abbr of ``\underline{M}ath \underline{M}atrix \underline{T}ransposed''.

\begin{tabular}{*{12}{l}}
    $\mmat$ & \lstinline`\mmat` & $\mmht$ & \lstinline`\mmht` & $\mmot$ & \lstinline`\mmot` & $\mmvt$ & \lstinline`\mmvt` & $\mmgammat$  & \lstinline`\mmgammat` & $\mmupsilont$ & \lstinline`\mmupsilont` \\
    $\mmbt$ & \lstinline`\mmbt` & $\mmit$ & \lstinline`\mmit` & $\mmpt$ & \lstinline`\mmpt` & $\mmwt$ & \lstinline`\mmwt` & $\mmdeltat$  & \lstinline`\mmdeltat` & $\mmphit$     & \lstinline`\mmphit` \\
    $\mmct$ & \lstinline`\mmct` & $\mmjt$ & \lstinline`\mmjt` & $\mmqt$ & \lstinline`\mmqt` & $\mmxt$ & \lstinline`\mmxt` & $\mmthetat$  & \lstinline`\mmthetat` & $\mmpsit$     & \lstinline`\mmpsit` \\
    $\mmdt$ & \lstinline`\mmdt` & $\mmkt$ & \lstinline`\mmkt` & $\mmrt$ & \lstinline`\mmrt` & $\mmyt$ & \lstinline`\mmyt` & $\mmlambdat$ & \lstinline`\mmlambdat` & $\mmomegat$   & \lstinline`\mmomegat` \\
    $\mmet$ & \lstinline`\mmet` & $\mmlt$ & \lstinline`\mmlt` & $\mmst$ & \lstinline`\mmst` & $\mmzt$ & \lstinline`\mmzt` & $\mmxit$     & \lstinline`\mmxit` &                                          \\
    $\mmft$ & \lstinline`\mmft` & $\mmmt$ & \lstinline`\mmmt` & $\mmtt$ & \lstinline`\mmtt` &         &                          & $\mmpit$     & \lstinline`\mmpit` &                                          \\
    $\mmgt$ & \lstinline`\mmgt` & $\mmnt$ & \lstinline`\mmnt` & $\mmut$ & \lstinline`\mmut` &         &                          & $\mmsigmat$  & \lstinline`\mmsigmat` &                                          \\
\end{tabular}
%------------------------------------------------------------------------------%


%------------------------------------------------------------------------------%
\subsection{Special Vector and Matrix Notations}
%------------------------------------------------------------------------------%
\begin{tabular}{*{6}{l}}
    $\mvzero$ & \lstinline`\mvzero` & $\mmzero$ & \lstinline`\mmzero` & $\mtzero$ & \lstinline`\mtzero` \\
    $\mvone$  & \lstinline`\mvone` & $\mmone$  & \lstinline`\mmone` & $\mtone$  & \lstinline`\mtone` \\
\end{tabular}
%------------------------------------------------------------------------------%



%------------------------------------------------------------------------------%
\section{Useful Functions and Operators}
%------------------------------------------------------------------------------%
\begin{tabular}{*{12}{l}}
    $\diff$   & \lstinline`\diff` & $\eig$  & \lstinline`\eig` & $\mean$ & \lstinline`\mean` & $\card$   & \lstinline`\card` & $\dist$       & \lstinline`\dist` & \\
    $\Diff$   & \lstinline`\Diff` & $\tr$   & \lstinline`\tr` & $\var$  & \lstinline`\var` & $\argmin$ & \lstinline`\argmin` & $\rot$        & \lstinline`\rot` & \\
    $\Expect$ & \lstinline`\Expect` & $\lcm$  & \lstinline`\lcm` & $\corr$ & \lstinline`\corr` & $\argmax$ & \lstinline`\argmax` & $\curl$       & \lstinline`\curl` & \\
    $\diag$   & \lstinline`\diag` & $\rand$ & \lstinline`\rand` & $\conv$ & \lstinline`\conv` & $\argopt$ & \lstinline`\argopt` & $\divergence$ & \lstinline`\divergence` & \\
\end{tabular}
%------------------------------------------------------------------------------%


%------------------------------------------------------------------------------%
\section{Useful Aliases and Generators}
%------------------------------------------------------------------------------%
\begin{itemize}
    \item \textbf{Derivatives.} Command: \lstinline`\[d]frac(diff|partial)(s|{var1}){var2}`. \lstinline`var1` and \lstinline`var2` is numerator and denominator, respectively. \lstinline`[d]` is just like the \lstinline`\dfrac` providing a display mode. \lstinline`(diff|partial)` provides derivative or partial derivative. \lstinline`(s|{var1})` means that the numerator is skippable. For example,

          \renewcommand{\arraystretch}{2}
          \begin{tabular}{*{4}{l}}
              \toprule
              Text                     & \TeX                     & Text                      & \TeX                     \\
              \midrule
              $\fracdiff{u}{x}$        & \lstinline`\fracdiff{u}{x}` & $\dfracdiff{u}{x}$        & \lstinline`\dfracdiff{u}{x}` \\
              $\fracdiff{^2u}{x^2}$    & \lstinline`\fracdiff{^2u}{x^2}` & $\dfracdiff{^2u}{x^2}$    & \lstinline`\dfracdiff{^2u}{x^2}` \\
              $\fracdiffs{x}$          & \lstinline`\fracdiffs{x}` & $\dfracdiffs{x}$          & \lstinline`\dfracdiffs{x}` \\
              $\fracpartial{u}{x}$     & \lstinline`\fracpartial{u}{x}` & $\dfracpartial{u}{x}$     & \lstinline`\dfracpartial{u}{x}` \\
              $\fracpartial{^2u}{x^2}$ & \lstinline`\fracpartial{^2u}{x^2}` & $\dfracpartial{^2u}{x^2}$ & \lstinline`\dfracpartial{^2u}{x^2}` \\
              $\fracpartials{x}$       & \lstinline`\fracpartials{x}` & $\dfracpartials{x}$       & \lstinline`\dfracpartials{x}` \\
              \bottomrule
          \end{tabular}
          \renewcommand{\arraystretch}{1.25}

    \item \textbf{Function vaules at exact point.} Command: \lstinline`\mfwhen{var1}{var2}`. \lstinline`var1` and \lstinline`var2` is function and point position, respectively. For example, \lstinline`\mfwhen{\fracpartial{u}{t}}{x=5}` gets $\mfwhen{\fracpartial{u}{t}}{x=5}$.

    \item \textbf{Auto sized brackets.} Command: \lstinline`\mclosure{}` for \lstinline`()`, \lstinline`\mclosuresquare{}` for \lstinline`[]`, \lstinline`\mclosurebrace{}` for \lstinline`{}`. For example, $\mclosurebrace{ \mclosuresquare{ \mclosure{a^2 + b^2}^2 }^3 }$.

    \item \textbf{Vector(Sequence) generator.} Command \lstinline`\mvct[z][t]{var1}{var2}`. \lstinline`var1` and \lstinline`var2` is variable name and the last index, respectively. The index is begin from 1 in default. \lstinline`[z]` makes index begins from 0. \lstinline`[t]` makes this vector transposed into a column vector. For example,

          \begin{tabular}{*{4}{l}}
              \toprule
              Text           & \TeX                     & Text            & \TeX                     \\
              \midrule
              $\mvct{a}{n}$  & \lstinline`\mvct{a}{n}` & $\mvctz{a}{n}$  & \lstinline`\mvctz{a}{n}` \\
              $\mvctt{a}{n}$ & \lstinline`\mvctt{a}{n}` & $\mvctzt{a}{n}$ & \lstinline`\mvctzt{a}{n}` \\
              \bottomrule
          \end{tabular}

    \item \textbf{A list of equations group by a brace.} Command \lstinline`\mequlist{...}`. Also provide environment \lstinline`equlist`, which is similar with the \lstinline`cases` environment. For example, \lstinline`\mequlist{x + y &= 10 \\ 4x + 2y &= 30}` gets $\mequlist{x + y &= 10 \\ 4x + 2y &= 30}$.
\end{itemize}
%------------------------------------------------------------------------------%
\end{document}